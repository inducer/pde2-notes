\documentclass{article}
\usepackage{geometry,amsmath,amssymb,enumerate,bbm,color}
\geometry{letterpaper}

\def\iintg{\iint_{\hat G}}
\def\dxdt{\;\mathd x\;\mathd t}
\newtheorem{theorem}{Theorem}[section]
\newtheorem{lemma}{Lemma}[section]
\newtheorem{definition}{Definition}[section]
\numberwithin{equation}{section}

%%%%%%%%%% Start TeXmacs macros
\newcommand{\mathd}{\mathrm{d}}
\newcommand{\op}[1]{#1}
\newcommand{\tmem}[1]{{\em #1\/}}
\newcommand{\tmname}[1]{\textsc{#1}}
\newcommand{\tmop}[1]{\ensuremath{\operatorname{#1}}}
\newcommand{\tmstrong}[1]{\textbf{#1}}
\newcommand{\tmtexttt}[1]{{\ttfamily{#1}}}
\newenvironment{enumeratealpha}{\begin{enumerate}[a{\textup{)}}] }{\end{enumerate}}
\newenvironment{enumeratealphacap}{\begin{enumerate}[A.] }{\end{enumerate}}
\newenvironment{proof}{\noindent\textbf{Proof\ }}{\hspace*{\fill}$\Box$\medskip}
\definecolor{grey}{rgb}{0.75,0.75,0.75}
\definecolor{orange}{rgb}{1.0,0.5,0.5}
\definecolor{brown}{rgb}{0.5,0.25,0.0}
\definecolor{pink}{rgb}{1.0,0.5,0.5}
%%%%%%%%%% End TeXmacs macros

\begin{document}

\title{On the Initial Value Problem for the Basic Equations of
Hydrodynamics}\author{Eberhard Hopf}\maketitle

{\tableofcontents}





Translation by Andreas Kl\"ockner, \tmtexttt{kloeckner@dam.brown.edu}. I would
like to hear about any errors or other comments you may have. {\color{green}
GREEN} text is loosely translated. {\color{red} RED} text marks spots where I
was unsure.

\section{Introduction.}

Let the points of $n$-dimensional space be designated by $x$, let $x_1, x_2,
\ldots, x_n$ be the coordiantes in a fixed cartesian coordinatesystem.
Further, let $\mathd x = \mathd x_1 \mathd x_2 \cdots \mathd x_n$ be the
volume element in $x$-space. Let $u (x, t)$ be a time-dependent vector field
defined on an open subset $\hat{G}$ of $x$-$t$-space with components $u_i$ in
the aforementioned coordinate system. We will not assume that $\hat{G}$ is
connected, and only for brevity will be speaking of {\color{red} regions}.
Regions in $x$-space will be denoted $G$, in $x$-$t$-space they will be
denoted $\hat{G}$. The fact that a vector field $u (x, t)$ which is
continuously $x$-differentiable on an $x$-$t$-region $\hat{G}$ is
divergence-free is characterized by the differential equation
\begin{equation}
  \tmop{div} u = \frac{\partial u}{\partial x_{\nu}} = 0,
\end{equation}
where we note that throughout this paper we will be using the common summation
convention without the use of the sum symbol. There is also another well-known
differential-less characterization of the fact that the divergence is zero. We
say that a scalar or vector-valued function $v (x, t)$ on $\hat{G}$ belongs to
{\tmem{class $N$ on $\hat{G}$}} iff $v \equiv 0$ outside a suitable compact
subset of this region. The functions of this class, which we will be referring
to often, thus vanish on a boundary strip of $\hat{G}$. This aforementioned
characterization is then: A field $u (x, t)$ which is continuously
$x$-differentiable on $\hat{G}$ is called {\tmem{divergence-free}} on
$\hat{G}$ iff
\begin{equation}
  \iintg u_i \frac{\partial h}{\partial x_i} \mathd x \mathd t = 0
\end{equation}
for any function $h (x, t)$ of class $N$ in $\hat{G}$ that is uniquely
determined and continuously $x$-differentiable on $\hat{G}$. This fact is a
consquence of Gauss' Theorem which is applicable because $h \in N$ in
$\hat{G}$ and because of the fundamental lemma of variational calculus. If we
introduce the the scalar product of two vector fields $v (x, t)$ and $w (x,
t)$ on $\hat{G}$ as
\[ \iintg v_i w_i \mathd x \mathd t, \]
we can say that ``a field $u$ which is continuously $x$-differentiable in
$\hat{G}$ is divergence-free in $\hat{G}$'' means that $u$ is orthogonal in
$\hat{G}$ to the gradient field of any function of class $N$ that is uniquely
determined and continuously $x$-differentiable in
$\hat{G}${\footnote{\label{fn:formulation-xt}The formulation of these terms in
$x$-$t$-space rather than just in $x$-space is advantageous for our problem.
Applications of Hilbert space theory can be found in the following works:
{\tmname{O.~Nikodym}}, Sur un th\'eor\`eme de M.S. Zaremba concernant les
fonctions harmoniques. J. Math. pur appl., Paris, S\'er. IX, {\tmstrong{12}}
(1933), 95--109; {\tmname{J.~Leray}}, Sur le mouvement d'un liquide visqueux
emplissant l'espace. Acta math., Uppsala {\tmstrong{63}} (1934), 193--248;
{\tmname{H.~Weyl}}, The method of orthogonal projection in potential theory.
Duke math J. {\tmstrong{7}} (1940), 411--444.}}.

The following counterpart of this fact is of interest here: It is necessary
and sufficient for a field $h' (x, t)$ which is continuous in $\hat{G}$ (and
which has components $h'_i$) to be the gradient field $h'_i = \partial h /
\partial x_i$ of an in $\hat{G}$ uniquely determined and continuously
$x$-differentiable function $h (x, t)$ that it is orthogonal in $\hat{G}$ to
any divergence-free field of class $N$ that is continuously $x$-differentiable
in $\hat{G}$.

Necessity is once more a consequence of the Integral Theorem. The following
considerations show sufficiency. The consideration of fields of the form $w
(x, t) = \varphi (t) \omega (x)$ with scalar $\varphi$ first shows that we may
constrain ourselves to the the corresponding claim for $x$-regions $G$. So,
assume
\[ \int_G w_i h_i' \mathd x = 0 \]
for any smooth divergence-free field $w (x)$ of class $N$ in $G$. The claim
follows if we can show that the circulation of the field $h'$
\[ \int_{\mathfrak{C}} h_i' \mathd x_i = \int_{\mathfrak{C}} h'_s \mathd s \]
vanishes along any closed path $\mathfrak{C}$ in $G$. It is easy to see that
this needs to be shown only for continuously curved paths without
self-intersections. We will obtain this vanishing through a suitable choice of
fields $w$. For any given small $\varepsilon > 0$, there is a vector field $w
(x)$ which is smooth and divergence-free in $G$ and which has the following
properties: $w$ is non-zero only in a closed tube around $\mathfrak{C}$ of
thickness $< \varepsilon$. On any plane tube section that cuts $\mathfrak{C}$
orthogonally, the vector $w$ forms an angle $< \varepsilon$ with the normal
direction (i.e. the direction of $\mathfrak{C}$ in the section). The sectional
flow of $w$, which is independent of the exact shape of the section because
$w$ is divergence-free, is equal to $1$. This fact suffices to prove the
vanishing of the circulation along $\mathfrak{C}$. We consider such a field $w
(x)$ that belongs to a given (but sufficiently small) $\varepsilon$. If we let
$\mathd F$ denote the hypersurface element on these tube sections and if we
choose the arc length $s$ along $\mathfrak{C}$ as the parameter transverse to
the sections, we can write the volume element $\mathd x$ in the tube as $\rho
(x) \mathd F \mathd s$, where we assume $\rho$ to be continuous in a
neighborhood of $\mathfrak{C}$ and equal to $1$ on $\mathfrak{C}$. Then
\[ \int h_i' w_i \mathd x = \int \left[ h'_w |w| \rho \mathd F \right] \mathd
   s. \]
If we replace the component $h'_w$ by the component $h'_s$ taken at the
intersection of $\mathfrak{C}$ with the section, $|w (x) |$ by the component
$w_s (x)$ taken in a direction normal to $\mathd F$ and $\rho$ by 1, then the
right-hand side integral becomes
\[ \int h'_s \left[ \int w_s \mathd F \right] \mathd s = \int h'_s \mathd s,
\]
i.e. the circulation. Based upon the aforementioned properties of the field
$w$, we can meanwhile easily prove that that the error introduced by these
replacements goes to zero with $\varepsilon$. Thereby the claim is proven.

The basic equations of Navier-Stokes for the movement of a homogeneous,
incompressible liquid are
\begin{equation}
  \frac{\partial u_i}{\partial t} + u_{\alpha} \frac{\partial u_i}{\partial
  x_{\alpha}} = - \frac{\partial p}{\partial x_i} + \mu \frac{\partial^2
  u_i}{\partial x_{\beta} \partial x_{\beta}},
\end{equation}
where $\mu$ is a positive constant, namely the {\tmem{kinematic viscosity
coefficient}} and
\[ \tmop{div} u = 0. \]
Let $u (x, t), p (x, t)$ be a solution in an $x$-$t$-region $\hat{G}$ which we
assume to be continuous along with all the occurring derivatives $u_t$, $u_x$,
$u_{x x}$. We will now introdcue a new time-dependent vector field $a = a (x,
t)$ which is divergence-free in $\hat{G}$. It is assumed to be of class $N$ in
$\hat{G}$ and sufficiently smooth: $a$ and the derivatives $a_t, a_x, a_{x x}$
should be continuous in $\hat{G}$. Otherwise, there will be no no requirements
on the field $a$. Because $a \in N$ in $G$ and because
\[ u_{\alpha} \frac{\partial u_i}{\partial x_{\alpha}} = \frac{\partial u_i
   u_{\alpha}}{\partial x_{\alpha}}, \]
we have
\begin{eqnarray*}
  \iintg a_i \frac{\partial u_i}{\partial t} \mathd x \mathd t & = & - \iintg
  \frac{\partial a_i}{\partial t} u_i \mathd x \mathd t,\\
  \iintg a_i u_{\alpha} \frac{\partial u_i}{\partial x_{\alpha}} \mathd x
  \mathd t & = & - \iintg \frac{\partial a_i}{\partial x_{\alpha}} u_{\alpha}
  u_i \mathd x \mathd t,\\
  \iintg a_i \frac{\partial^2 u_i}{\partial x_{\beta} \partial x_{\beta}}
  \mathd x \mathd t & = & - \iintg \frac{\partial a_i}{\partial x_{\beta}}
  \frac{\partial u_i}{\partial x_{\beta}} \mathd x \mathd t = \iintg
  \frac{\partial^2 a_i}{\partial x_{\beta} \partial x_{\beta}} \mathd x \mathd
  t
\end{eqnarray*}
and since $\tmop{div} a = 0$ and $a \in N$, we have
\[ \iintg a_i \frac{\partial p}{\partial x_i} \mathd x \mathd t = 0. \]
Thereby we find that the field $u (x, t)$ satisfies the following condition
\begin{equation}
  \label{eq:ns-weak} \iintg \frac{\partial a_i}{\partial t} u_i \mathd x
  \mathd t + \iintg \frac{\partial a_i}{\partial x_{\alpha}} u_{\alpha} u_i
  \mathd x \mathd t + \mu \iintg \frac{\partial^2 a_i}{\partial x_{\beta}
  \partial x_{\beta}} u_i \mathd x \mathd t = 0
\end{equation}
for any sufficiently smooth field $a (x, t)$ on $\hat{G}$ with the properties
\begin{equation}
  \label{eq:a-requirements} \tmop{div} a = 0 \text{in $\hat{G}$}, \hspace{1em}
  a \in N \text{in $\hat{G}$} .
\end{equation}
In addition, we need to take into account that $u$ is divergence-free, i.e.
\begin{equation}
  \label{eq:divfree-weak} \iintg \frac{\partial h}{\partial x_i} u_i \mathd x
  \mathd t = 0, \hspace{1em} h \in N \text{in}  \hat{G}
\end{equation}
holds for any function of the mentioned class that is sufficiently smooth on
$\hat{G}$. We have thereby reduced the basic equations to the form of
equations between linear functional operators of arbitrary fields and
functions $a$ and $h$. The essential part of this is that the unknown field
$u$ on which these operators depend occurs without any derivatives.

We still need to convince ourselves that we may revert from equations
(\ref{eq:ns-weak}) and (\ref{eq:divfree-weak}) to the differential form of the
equations if we restrict ourselves to sufficiently smooth solution fields $u$
in $\hat{G}$. We already know that under this assumption
(\ref{eq:divfree-weak}) goes back to $\tmop{div} u = 0$ in $\hat{G}$. For a
sufficiently smooth $u$, we may undo all the integrations-by-parts. It follows
that
\[ \iintg a_i \left\{ \frac{\partial u_i}{\partial t} + u_{\alpha}
   \frac{\partial u_i}{\partial x_{\alpha}} - \mu \frac{\partial^2
   u_i}{\partial x_{\beta} \partial x_{\beta}} \right\} \mathd x \mathd t \]
must hold for any sufficiently smooth field $a (x, t)$ of the form
(\ref{eq:a-requirements}). Using the theorem proved above, we may conclude
that the curly braces must be the partial derivatives of a uniquely determined
function $p (x, t)$, i.e. that the differential equations of motion must hold
in $\hat{G}$. We see that the above integral form of the equations exactly
expresses the physical demand that the pressure be unique.

It is quite natural to build the general mathematical theory on the integral
form of the equations. But then it is appropriate to rid ourselves of the
artificial restriction to smooth solution fields $u$. The occurrence of the
quadratic forms
\[ \int u_i u_i \mathd x, \hspace{1em} \int \frac{\partial u_i}{\partial
   x_{\beta}}  \frac{\partial u_i}{\partial x_{\beta}} \mathd x \]
in the energy equation leads us to base the problem on a Hilbert space of
vector fields. It is a methodical advantage that in this broader framework the
differentiability properties of the solutions $u$ become the subject of a
problem that can be entirely separated from the problem of
existence.{\footnote{Compare the treatment of quadratic variation and linear
differential problems by methods of Hilbert spaces in {\tmname{R.~Courant}}
and {\tmname{D.~Hilbert}}, Methoden der mathematischen Physik, Volume 2,
Berlin 1937, Chapter VII.}}

The common initial value problem of the basic equations of hydrodynamics is
the following: We need to find the solution $u (x, t)$ in a prescribed, moving
region $G (t)$ ($t \geqslant 0$) of $x$-space, while $u (0)$ in $G (0)$ is
prescribed (together with a suitably formulated condition of continuous
continuation for $t \rightarrow 0$) and the boundary values at the boundary of
$G (t)$, $t > 0$ are also given (with a suitably formulated sense of
continuation). {\tmname{J.~Leray}} dedicated three sizable works to this
problem in the early thirties{\footnote{{\tmname{J.~Leray}}, a) \'Etude de
diverses \'equations int\'egrales non lin\'eaires et de quelques probl\`emes
que pose l'Hydrodyamique. J.Math.pur.appl. Paris, S\'er. IX {\tmstrong{12}}
(1933) 1--82; b) Essay sur les mouvements plans d'un liquide visqueux que
limitent des parois. c) loc. cit. in footnote \ref{fn:formulation-xt}.}}.
These inquiries had already forced Leray to use the methods of Hilbert space
and the integral interpretation of the equations in three
dimensions{\footnote{A long while before then, {\tmname{C.W.~Oseen}} had based
his well-known hydrodynamic inquiries on a form of the basic equations that is
free of second derivatives. However, he only succeeded in proving existence
for sufficiently small times. Cf. his work Hydrodynamik (Leipzig 1927)}}. In
his works, Leray solved the question of existence for all $t > 0$ in the
following cases, a) $G =\mathbbm{R}^2$ under the added condition of finite
kinetic energy, b) $G$ is a fixed oval with zero boundary values, c) $G
=\mathbbm{R}^3$ under the added condition of finite kinetic energy. The
remarkable analysis that Leray dedicates to the question of differentiability
point to a strange difference between the dimensions $n = 2$ and $n > 2$.
While, at least if in the first case $G$ is the entire plane, the proof of
infinite differentiability is successful, the proof methods that one should
view as natural fail for $n \geqslant 3$. Even for arbitrary smoothness of all
prescribed data, the proof of smoothness of the solution did not work out. The
other strange thing is the failure of the uniqeness proff in three dimensions.
These questions are still not answered satisfactorily. It is hard to believe
that the initial value problem of viscous liquids for $n = 3$ should have more
than one solution, and more attention should be paid to the settling of the
uniqueness question. However, newer research indicates that for nonlinear
partial differential problems the number of independent variables has
significant influence on the local properties of solutions.

In the present work, which is also dedicated to the initial value problem and
in which we assume the integral view of the equations as their primary form,
we will leave aside the questions of differentiability and uniqueness. We hope
to come back to these things as well as to the proof of the energy equation
(which is easy in our context) in later memoranda. The main point of this work
is that the construction of approximate solutions that takes such broad space
in Leray's work is replaced here by simpler process, which may also be applied
to a much broader classes of partial differential problems. We also hope to
come back to this issue later. This method enables the solution of the initial
value problem for all $t > 0$ in substantial generality, however in this first
memorandum what matters to us is more the exposition of the basic idea of the
method rather than the generality of the results. We will restrain ourselves
to the case that the $x$-region $G$ is fixed in time, but otherwise completely
arbitrary, and where $u$ has vanishing boundary values. The boundary condition
will be defined in terms of Hilbert space--broad enough to guarantee
solvability, and narrow enough to guarantee the uniqueness of the solution, at
least in two dimensions{\footnote{\label{fn:bc-finite-kinetic}If $G$ is the
entire $x$-space, the boundary condition thus phrased becomes the condition of
finite kinetic energy and finite dissipation integral.

The phrasing of the boundary condition is suggested by the work of
{\tmname{R.~Courant}} and {\tmname{D.~Hilbert}}, Methoden der mathematischen
Physik, Vol. 2, Berlin 1937, Chap. VII, �1, 3rd section.}}. In pure existence
theory, the number of space dimensions will not play any role.

\section{The Function Class $H'$. Solutions of Class $H'$.}

We will take the class $H$ with respect to an $x$-$t$-region $\hat{G}$ to mean
the class of all real, measurable functions $f (x, t)$ defined on this region
with finite norm
\[ \iintg f^2 \mathd x \mathd t. \]
$H$ is a real Hilbert space. Terms such as weak and strong convergence in
$\hat{G}$ will be understood in the following with respect to the norm. We
remind that a sequence of functions $f \in H$ in $\hat{G}$ converges weakly if
first, the norms of all $f$ remain below a fixed value and second, if
\[ \iintg fg \mathd x \mathd t \rightarrow \iintg f^{\ast} g \mathd x \mathd t
\]
holds for any fixed function $g \in H$ in $\hat{G}$. While maintaining the
first condition, the second one may be weakened to the effect that the
sequence of numbers
\[ \iintg fg \mathd x \mathd t \]
converges for any fixed $g$ in a set that is strongly dense in $H$. Then,
there exists one, and essentially only one weak limit function $f^{\ast}$ in
$\hat{G}$. Besides these terms, for which we have assumed an $x$-$t$-region,
we will have to use the same terms for a purely spatial $x$-region $G$. In
this case, we will base our considerations on the norm
\[ \int_G f^2 \mathd x. \]
We remind the reader of the weak compactness of a sequence of functions with
uniformly bounded norms (F.~Riesz's Theorem). The following criterion for
strong convergence, which was also used extensively by Leray, will also be
necessary here. For a sequence of functions that converges weakly in $\hat{G}$
to a limit function $f^{\ast}$, we have
\[ \overline{\op{\lim}} \iintg f^2 \mathd x \mathd t \geqslant \iintg
   (f^{\ast})^2 \mathd x \mathd t, \]
where equality holds if and only if $f \rightarrow f^{\ast}$ in the strong
sense. All these things transfer to vector fields $u$, $v$ on $\hat{G}$ if we
use the scalar product
\[ \iintg u_i v_i \mathd x \mathd t \]
and the corresponding norm.

\begin{lemma}
  \label{lem:l2-lsc}If the vector fields $u (x, t)$ converge weakly in
  $\hat{G}$ to a limit field $u^{\ast} (x, t)$, then
  \[ \overline{\lim} \iintg u_i u_i \mathd x \mathd t \geqslant \iintg
     u_i^{\ast} u_i^{\ast} \mathd x \mathd t. \]
  Equality holds if and only if the convergence in $\hat{G}$ is strong.
\end{lemma}

Like Leray, we need the term of a generalized (purely spatial) $x$-derivative
of functions $f (x, t)$ and fields $u (x, t)$.

\begin{definition}
  \label{def:hprime}An $f (x, t)$ defined on an $x$-$t$-region $\hat{G}$ is
  defined to belong to the class $H'$ if and only if it has the following
  properties: $f$ belongs to $H$ in $\hat{G}$. There exist $n$ functions $f_{'
  i}$ belonging to $H$ in $\hat{G}$ such that the relations
  \begin{equation}
    \label{eq:def-weak-deriv} \iintg hf_{' i} \mathd x \mathd t = - \iintg
    \frac{\partial h}{\partial x_i} f \mathd x \mathd t \hspace{1em} (h \in N
    \text{in}  \hat{G})
  \end{equation}
  are satisfied for any function $h (x, t)$ which is continuous in $\hat{G}$
  along with its derivatives and which belongs to class $N$, and for any $i =
  1, 2, \ldots, n$.
\end{definition}

The class $H'$ obviously contains any $f (x, t)$ that is continuously
$x$-differentiable in $\hat{G}$ such that $f$ and all $\partial f / \partial
x_i$ belong to $H$ in $\hat{G}$. For such an $f$, we have $\partial f /
\partial x_i = f_{' i}$. This follows from the integral theorem and the demand
that $h$ must belong to $N$, i.e. that $h$ vanishes outside a certain compact
subset of $\hat{G}$. Obviously, generalized $x$-derivatives $f_{' i}$ in $G$
are uniquely determined except for the values on an $x$-$t$-zero set in the
case of $f \in H'$ in $\hat{G}$.

\begin{lemma}
  \label{lem:weak-deriv-weak-conv}If a sequence of functions of class $H'$
  converge weakly to $f^{\ast}$ and for all $f$ the expressions
  \[ \iintg f^2 \mathd x \mathd t + \iintg f_{' i} f_{' i} \mathd x \mathd t
  \]
  are uniformly bounded, then $f^{\ast}$ also belongs to $H'$ in $\hat{G}$ and
  every $x$-derivatives $f_{' i}$ converges weakly to the corresponding
  $x$-derivative $f^{\ast}_{' i}$.
\end{lemma}

\begin{proof}
  Every $f$ satisfies (\ref{eq:def-weak-deriv}), where $h$ is an arbitrary
  function that is admissible there. The right hand sides converge to
  \[ - \iintg \frac{\partial h}{\partial x_i} f^{\ast} \mathd x \mathd t. \]
  For a fixed $h$ and $i$, the left hand sides converge along the sequence of
  the $f$'s. The admissible functions $h$ in $\hat{G}$ lie strongly dense in
  the Hilbert space $H$.Thus, for any fixed $i$ the sequence of the $f_{' i}$
  is weakly convergent. If we let $f^{\ast}_i$ denote the limit function, then
  from (\ref{eq:def-weak-deriv}), we conclude that
  \[ \iintg hf^{\ast}_i \mathd x \mathd t = - \iintg \frac{\partial
     h}{\partial x_i} f^{\ast} \mathd x \mathd t \]
  holds for any admissible $h$ and $i$. By Definition \ref{def:hprime},
  $f^{\ast}$ belongs to $H'$ in $\hat{G}$, and because of uniquness of the
  $x$-derivative, we have $f^{\ast}_i = f^{\ast}_{' i}$.
\end{proof}

A field is said to be of class $H'$ in $\hat{G}$ if this is the case for all
components.

In the above integral form of the basic equations of hydrodynamics, there are
no derivatives on $u$. It is however practical to make a weak
differentiability assumption like membership in the class $H'$ on the
solutions $u$. We may then write for the friction term in (\ref{eq:ns-weak})
\begin{equation}
  \mu \iintg \frac{\partial^2 a_i}{\partial x_{\beta} \partial x_{\beta}} u_i
  \dxdt = - \mu \iintg \frac{\partial a_i}{\partial x_{\beta}} u_{i, \beta}
  \dxdt .
\end{equation}
\begin{definition}
  \label{def:ns-weak-solution}A field $u (x, t)$ is called a solution of class
  $H'$ of the basic equations of hydrodynamics in the $x$-$t$-region $\hat{G}$
  if it satisfies the following conditions:
  \begin{enumeratealpha}
    \item $u \in H'$ in $\hat{G}$.
    
    \item Vanishing divergence; any function $h$ which is of class $N$ in
    $\hat{G}$ and continuously $x$-differentiable satisfies the relation
    (\ref{eq:divfree-weak}).
    
    \item Equations of motion; any field $a (x, t)$ that is of class $N$ in
    $\hat{G}$, divergence-free and continuous along with its derivatives
    $a_t$, $a_x$, $a_{x x}$ satisfies the relation (\ref{eq:ns-weak}).
  \end{enumeratealpha}
\end{definition}

Observe that under the condition a) the term in the basic equations
(\ref{eq:ns-weak}) which is nonlinear in $u$ is a valid Lebesgue integral for
any admissible field $a$. That is already the case if $u \in H$ in $\hat{G}$.

Because of a) the condition of incompressibility b) is equivalent with
\[ \tmop{div} u \equiv u_{i, i} = 0 \]
for a.e. $(x, t) \in \hat{G}$.

We will consider all integrands in the basic equations {\eqref{eq:ns-weak}}
outside of $\hat{G}$ defined to zero. The integrals can then be extended over
all $x$-$t$-space. With this convention, the following theorem, which we would
like to prove here even though it is not needed in this paper, holds:

\begin{theorem}
  A solution of class $H'$ satisfies the equation
  \begin{equation}
    \label{eq:up-to-tau} \int_{t = \tau} a_i u_i \mathd x = \int \int_{t <
    \tau} \frac{\partial a_i}{\partial t} u_i \mathd x \mathd t + \int \int_{t
    < \tau} \frac{\partial a_i}{\partial x_{\alpha}} u_{\alpha} u_i \mathd x
    \mathd t - \mu \int \int_{t < \tau} \frac{\partial a_i}{\partial
    x_{\beta}} u_{i, \beta} \dxdt
  \end{equation}
  for a.e. value of $\tau$.
\end{theorem}

\begin{proof}
  Consider that along with $a (x, t)$, $h (t) a (x, t)$ is also an admissible
  field if $h (t)$ is an arbitrary continuously differentiable function for
  all $t$. If we replace $a$ by $h a$ in equation {\eqref{eq:ns-weak}}, which
  we write abbreviated as
  \[ \int \int K [a, u] \mathd x \mathd t = \int_{- \infty}^{\infty} \left\{
     \int_{t = \tau} K [a, u] \right\} \mathd \tau = 0, \]
  it follows that the equation
  \begin{equation}
    \label{eq:tau-a-ha} \int_{- \infty}^{\infty} h (\tau) \left\{ \int_{t =
    \tau} K \mathd x \right\} \mathd \tau + \int_{- \infty}^{\infty} h' (\tau)
    \left\{ \int_{t = \tau} a_i u_i \mathd x \right\} \mathd \tau = 0
  \end{equation}
  is also satisfied. The terms in curly braces are Lebesgue-integrable
  functions of $\tau$ on $- \infty < \tau < \infty$ that vanish for all large
  $| \tau |$. The validity of {\eqref{eq:tau-a-ha}} for abritray $h (\tau)$
  with continuous $h' (\tau)$ is equivalent with the fact that
  \[ \int_{t = \tau} a_i u_i \mathd x = \int_{- \infty}^t \left\{ \int_{t
     \text{fixed}} K \mathd x \right\} \mathd t = \int \int_{t < \tau} K \dxdt
  \]
  for a.e. $\tau$.
\end{proof}

In {\eqref{eq:up-to-tau}}, the left hand side is defined for just a.e. $\tau$,
while the right hand side is an {\color{red} absolutely continuous}
(totalstetig) function of $\tau$. In fact, one can prove: A solution of lcass
$H'$ in $\hat{G}$ can be changed on an $x$-$t$-zero set such that the new $u$
satisfies {\eqref{eq:up-to-tau}} without exception, i.e. for any admissible
$a$ and any $\tau$. But we will not elaborate further on this here.

\section{The Boundary Condition of Vanishing. The Initial Value Problem.}

\label{sec:bc-vanish-ivp}The cross sections $t = \tmop{const}$ of the
$x$-$t$-region $\hat{G}$ are $x$-region $G (t)$. By using just terms of
Hilbert space, we need to get as close as possible to the boundary condition
of vanishing of a function $g (x, t)$ and a field $u (x, t)$ for all $t$ on
the boundary of $G (t)$. This can be achieved by obtaining the function $g$
from function of class $N$ in $\hat{G}$ by means of a limit process. In doing
so, it is necessary to use sufficiently effective bounds on the spatial
$x$-derivatives (but not on the $t$-derivatives) of the approximating
functions, so that the ``vanishing'' remains intact along the boundaries of
the $x$-regions $G (t)$. We express the boundary condition by membership in
the following function class $H' (N)$.

\begin{definition}
  \label{def:bc-vanishing}A function $g (x, t)$ is said to be of class $H'
  (N)$ in $\hat{G}$ if it is a weak limit function in $\hat{G}$ of a sequence
  of functions $\gamma (x, t)$, which belong to $N$ in $\hat{G}$ and are
  continuous along with their $x$-derivatives und for which the expressions
  \begin{equation}
    \label{eq:bc-def-h1-norm} \iintg \gamma^2 \dxdt + \iintg \gamma_{' i}
    \gamma_{' i} \dxdt
  \end{equation}
  are uniformly bounded.{\footnote{Cf. {\tmname{Courant-Hilbert}}, l.c.
  footnote \ref{fn:bc-finite-kinetic}, p. 218. The definition of the boundary
  condition of vanishing given there is only seemingly stronger than ours. By
  S. Saks' Theorem the sequence of arithmetic means of a weakly convergent
  sequence has a strongly convergent subsequence. It follows from this theorem
  and from Lemma \ref{lem:weak-deriv-weak-conv} that for any $g$ in $H' (N)$,
  there exists a sequence of functions $\gamma$ of the above-mentioned kind
  such that
  \[ \gamma \rightarrow g, \hspace{1em} \gamma_{' i} \rightarrow g_{' i} \]
  holds inthe strong sense.}}
\end{definition}

It follows from Lemma \ref{lem:weak-deriv-weak-conv} that for a given
$x$-$t$-region $G$ the class $H' (N)$ is contained in the class $H'$.

\begin{lemma}
  \label{lem:cylinder-limit-hprime}Let $\hat{G}$ by a cylinder set $x \subset
  G$, $0 < t < T$. Let $g (x, t)$ be the weak limit in $\hat{G}$ of a sequence
  of functions $\gamma (x, t)$ continuously $x$-differentiable in $\hat{G}$
  that are of the following kind: For each $\gamma$ there is a compact subset
  of the $x$-region $G$ such that $\gamma$ vanishes for $x$ outside that set;
  \ also let the integrals {\eqref{eq:bc-def-h1-norm}} be uniformly bounded.
  Then $g$ belongs to $H' (N)$ in $\hat{G}$.{\footnote{If $G$ is all
  $x$-space, the class $H' (N)$ coincides with the class $H'$. In this case
  the admissible $\gamma$ are strongly dense in the function space $H'$ in the
  sense of the norm {\eqref{eq:bc-def-h1-norm}}.}}
\end{lemma}

\begin{proof}
  Observe the difference between the class of the $\gamma$ admissible in this
  lemma and the narrower class of the $\gamma$ of Definition
  \ref{def:bc-vanishing}. Membership of $\gamma$ in $N$ in the $x$-$t$-region
  $\hat{G}$ in the present case requires that $\gamma$ vanishes sufficiently
  close to $t = 0$ and $t = T$. But since only $x$-derivatives occur in
  {\eqref{eq:bc-def-h1-norm}}, this difference is inconsequential. If we
  replace the present $\gamma$ by functions $\varphi (t) \gamma (x, t)$, where
  $\varphi$ is continuous in {\color{red} $[0, T$]} and
  \[ \varphi = \left\{\begin{array}{ll}
       0 & \text{for} 0 < t < \varepsilon, \; T - \varepsilon < t < T,\\
       1 & \text{for} 2 \varepsilon < t < T - 2 \varepsilon,
     \end{array}\right. \]
  and otherwise $0 < \varphi < 1$ ($\varepsilon \rightarrow 0$), then
  Definition \ref{def:bc-vanishing} applies to the new $\tilde{\gamma} =
  \varphi \gamma$. Thus $g$ belongs to $H' (N)$.
\end{proof}

\begin{lemma}
  The relations
  \[ \iintg g_{' i} \dxdt = - \iintg gf_{' i} \dxdt \hspace{1em} (i = 1, 2,
     \ldots, n) \]
  are satisfied by any f of class $H'$ in $\hat{G}$ and any $g$ of class $H'
  (N)$ in $\hat{G}$.
\end{lemma}

\begin{proof}
  By Definition \ref{def:hprime}, the relations hold for any specified $f$ and
  for any $\gamma$ that is continuously $x$-differentiable and of class $N$ in
  $\hat{G}$. By Definition \ref{def:bc-vanishing}, $g$ is a weak limit of a
  sequence of such $\gamma$ with uniforml bounded integrals
  {\eqref{eq:bc-def-h1-norm}} By Lemma \ref{lem:weak-deriv-weak-conv}, besides
  $\gamma \rightarrow g$, we also have $\gamma_{' i} \rightarrow g_{' i}$
  weakly in $\hat{G}$. The relations that hold for $f$, $\gamma$ thus also
  hold for $f$, $g$.
\end{proof}

To facilitate a more convenient phrasing of the initial condition, we also
introduc the class $H (N)$. In doing so, we restrict ourselves to $x$-space
and field $u (x)$ that are defined in an $x$-region $G$. If we only consider
functions $f (x)$ that belong to both the classes $H$ and $N$, then it is
clear that the strong closure of these sets of functions is identical to $H$.
The same is true of vector fields in $G$. However, a difference arises if we
restrict ourselves to divergence-free fields in $G$.

\begin{definition}
  \label{def:h-n}A divergence-free field in $G$ of class $H$ is said to be of
  class $H (N)$ if it is a weak limit field of fields that belong to $N$ in
  $G$, that are twice continuously differentiable and that are
  divergence-free.{\footnote{By Saks' Theorem, it is then also the strong
  limit field of just these fields.}}
\end{definition}

One easily proves the following: If the field $u (x)$ is divergence-free and
of class $H (N)$ and if the function $\varphi (x)$ is of class $H'$, then
\[ \int_G u_i \varphi_{' i} \mathd x = 0. \]
Membership of a divergence-free field in $H (N)$ obviously replaces the
boundary condition of vanishing on the normal component.

We may now state the existence theorem for the hydrodynamic initial value
problem.

\begin{theorem}
  \label{the:existence}{\tmem{(Existence theorem)}}Let $G$ be an arbitrary
  region of $x$-space. Let the field $U (x)$ be divergence-free in $G$ and of
  class $H (N)$, but otherwise arbitrary. Then there is a field $u (x, t)$
  defined for all $t > 0$ in $G$ with the following properties:
  \begin{enumeratealphacap}
    \item In any $x$-$t$-cylinder region $x \subset G$, $0 < t < T$, $u$ is a
    solution of class $H'$ of the basic equations of hydrodynamics (cf.
    Definition \ref{def:ns-weak-solution}).
    
    \item ``Vanishing of the boundary values'' for $t > 0$: In any of the
    above-mentioned cylinder regions, $u$ belongs to $H' (N)$.
    
    \item Initial condition: For $t \rightarrow 0$, $u (x, t) \rightarrow U
    (x)$ converges strongly in $G$.
  \end{enumeratealphacap}
\end{theorem}

\section{Simplification of the Problem. The Approximation Procedure.}

\label{sec:simplification-approximation}For the construction of the solution
of the initial value problem for an $x$-region $G$ constant in time, we start
with the equation
\begin{equation}
  \label{eq:ns-tau-to-tauprime} \int_G a_i u_i \mathd x|_{t = \tau'} - \int_G
  a_i u_i |_{t = \tau} = \int_{\tau}^{\tau'} \int_G \frac{\partial
  a_i}{\partial t} u_i \dxdt + \int_{\tau}^{\tau'} \int_G \frac{\partial
  a_i}{\partial x_{\alpha}} u_{\alpha} u_i \dxdt + \mu \int_{\tau}^{\tau'}
  \int_G \frac{\partial^2 a_i}{\partial x_{\beta} \partial x_{\beta}} u \dxdt
  .
\end{equation}
\begin{lemma}
  Let the field $u (x, t)$ be defined in $G$ for all $t > 0$ and let it belong
  to class $H$ in any cylinder section $x \subset G$, $0 < t < T$ of
  $x$-$t$-space. Let it satisfy Equation {\eqref{eq:ns-tau-to-tauprime}} for
  all $\tau' > \tau > 0$ and for any field $a$ such that: $a = a (x)$ is twice
  continuously differentiable and
  \begin{equation}
    \label{eq:cylinder-admissible-testfield} a = a (x), \hspace{1em}
    \tmop{div} a = 0 \text{in} G, \hspace{1em} a \in N \text{in} G,
  \end{equation}
  i.e. $a (x)$ vansishes outside a suitable compact subset of $G$.
  
  Then $u$ satisfies the basic equation {\eqref{eq:ns-weak}} for the half
  cylinder $\hat{G}$: $x \subset G$, $t > 0$ and for any field admissible
  there (cf. condition c) in the definition \ref{def:ns-weak-solution} of a
  weak solution).
\end{lemma}

\begin{proof}
  If we write {\eqref{eq:ns-tau-to-tauprime}} in the abbreviated form
  \[ f (\tau') - f (\tau) = \int_{\tau}^{\tau'} g (t), \]
  we see that the equation
  \[ \int_0^{\infty} \varphi' (t) f (t) \mathd t + \int_0^{\infty} \varphi (t)
     g (t) \mathd t = 0 \]
  must be satisfied for any $\varphi$ that is continuously differentiable in
  $(0, \infty$) and which vanishes for all sufficiently small and large $t$.
  If we once more write the equation out in full, we recognize that Equation
  {\eqref{eq:ns-weak}} is satisfied in said half cylinder by any filed $a =
  \varphi (t) a (x)$, where $a (x)$ is an arbitrary one of the fields
  permitted above {\eqref{eq:cylinder-admissible-testfield}} and $\varphi (t)$
  is an arbitrary one of the functions permitted above. But now any $a (x, t)$
  permitted by condition c) in the definition \ref{def:ns-weak-solution} of a
  solution may be approximated in the half cylinder $\hat{G}$ by sums of
  fields of such special shape that in the basic equation {\eqref{eq:ns-weak}}
  integration and limit may be interchanged. E.g. one could always arrange
  that the convergnece of the fields and their derivatives up to a prescribed
  order in $\hat{G}$ is uniform and that the approximating fields all vanish
  outside a fixed compact subset of $\hat{G}$.
  
  It is thereby clear that a field $u (x, t)$ which satisfies
  {\eqref{eq:ns-tau-to-tauprime}} to the extent specified in the lemma, and
  which is further divergence-free and which belongs to class $H'$ in any
  cylinder section satisfies the full scope of the definition
  \ref{def:ns-weak-solution} of a solution on any cylinder section.
\end{proof}

The following fact yields an even better basic equation:

\begin{lemma}
  \label{lem:testfield-approx}There is a sequence of twice continuously
  differentiable and linearly independent fields in $G$ in the field space
  {\eqref{eq:cylinder-admissible-testfield}}
  \begin{equation}
    \label{eq:testfield-approx} a = a^{\nu} (x), \hspace{1em} \tmop{div}
    a^{\nu} = 0 \text{ in } G, \hspace{1em} a^{\nu} \in N \text{ in } G
  \end{equation}
  with the following property: An arbitrary twice continuously differentiable
  field in $G$ of the form {\eqref{eq:cylinder-admissible-testfield}} is the
  uniform limit field in $G$ of a sequence of finite linear combinations of
  the field $a^{\nu} (x)$, with uniform convergence of even the derivatives up
  to second order in $G$. For a given $a (x)$, only such linear combinations
  occur in this approximation that have the value zero outside a certain
  compact subset of $G$ which only depends on $a$.
\end{lemma}

Based upon this fact it is clear that a field $u (x, t)$ which is of class $H$
in each cylinder section and which satisfies the basic equation
{\eqref{eq:ns-tau-to-tauprime}} for all $\tau' > \tau > 0$ and for any field
$a$ of the mentioned sequence automatically does the same for all fields
{\eqref{eq:cylinder-admissible-testfield}} admitted above. In summary, we can
say that the basic equations {\eqref{eq:ns-weak}} can be replaced in their
entirety by the equations {\eqref{eq:ns-tau-to-tauprime}} with
{\eqref{eq:testfield-approx}}.

In the function sapce of divergence-free vector fields $a$,
{\eqref{eq:ns-tau-to-tauprime}}, {\eqref{eq:testfield-approx}} is an affine
coordinate representation of the basic equations of hydrodynamics. The affine
system of coordinate vectors {\eqref{eq:testfield-approx}} can, by means of a
unique linear transformation of a simple kind, be transformed into a new one
which is orthonormal in the sense of the bilinear form
\[ \int_G v_i w_i \mathd x. \]
We may additionally assume that the sequence {\eqref{eq:testfield-approx}}
satisfies this condition:
\begin{equation}
  \label{eq:testfield-orthogonality} \int_G a_i^{\lambda} a_i^{\nu} \mathd x =
  \delta_{\lambda, \nu} .
\end{equation}
\begin{lemma}
  \label{lem:testfields-completeness}The orthonormal system of the fields
  $a^{\nu} (x)$ is complete in the field space of divergence-free fields $U
  (x)$ of class $H (N)$ in $G$.
\end{lemma}

The proof results from Definition \ref{def:h-n} and Lemma
\ref{lem:testfield-approx}.

{\tmem{The Approximation Procedure}}. The $k$th approximation step consists
simply of only considering the first $k$ out of the infinitely many basic
equations {\eqref{eq:ns-tau-to-tauprime}}, {\eqref{eq:testfield-approx}},
\begin{equation}
  \label{eq:first-k-testfields} a = a^{\nu} (x) \hspace{1em} (\nu = 1, 2,
  \ldots, k)
\end{equation}
and trying to solve those through the ansatz
\begin{equation}
  \label{eq:first-k-ansatz} u = u^k (x, t) = \sum_{\nu = 1}^k \lambda_{\nu}
  (t) a^{\nu} (x)
\end{equation}
with as yet undetermined scalar factors $\lambda_{\nu} = \lambda_{\nu}^k$.
This ansatz automatically satisfies the condition of freedom from divergence
(because of {\eqref{eq:testfield-approx}}) and the boundary condition of
vanishing:
\begin{equation}
  \label{eq:first-k-divfree} \tmop{div} u^k = 0 \text{ in } G, \hspace{1em}
  u^k \in N \text{ in } G.
\end{equation}
Since only differentiable $\lambda (t)$ need to be considered and since the
admissible fields $a$ do not depend on $t$, the first $k$ equations
{\eqref{eq:ns-tau-to-tauprime}} may be written in the form
\begin{equation}
  \label{eq:ns-weak-first-k} \int_G a_i \frac{\partial u_i}{\partial t} \mathd
  x = \int_G \frac{\partial a_i}{\partial x_{\alpha}} u_{\alpha} u_i \mathd x
  + \mu \int_G \frac{\partial^2 a_i}{\partial x_{\beta} \partial x_{\beta}}
  u_i \mathd x.
\end{equation}
Because of {\eqref{eq:testfield-orthogonality}}, the $k$ equations
{\eqref{eq:ns-weak-first-k}}, {\eqref{eq:first-k-testfields}} together with
{\eqref{eq:first-k-ansatz}} represent a system of ordinary differential
equations
\begin{equation}
  \label{eq:first-k-ode} \frac{\mathd \lambda_{\nu}}{\mathd t} = F_{\nu}
  (\lambda_1, \ldots, \lambda_k) \hspace{1em} (\nu = 1, 2, \ldots, k)
\end{equation}
for the $\lambda$, in which the right hand sides $F_{\nu} = F_{\nu}^k$ are
polynomials in $\lambda$ with constant coefficients. The equations
{\eqref{eq:ns-weak-first-k}}, {\eqref{eq:first-k-testfields}},
{\eqref{eq:first-k-ansatz}} or the equivalent equations
{\eqref{eq:first-k-ode}} share with the strict hydrodynamic equations the
important property that for their solutions, the energy equation
\begin{equation}
  \label{eq:energy-equation} \frac{\mathd}{\mathd t}  \frac{1}{2} \int_G u_i
  u_i \mathd x = - \mu \int_G \frac{\partial u_i}{\partial x_{\beta}} 
  \frac{\partial u_i}{\partial x_{\beta}} \mathd x
\end{equation}
holds. Namely, since the equations {\eqref{eq:ns-weak-first-k}} hold for all
fields {\eqref{eq:first-k-testfields}}, they also hold for their linear
combinations {\eqref{eq:first-k-ansatz}} $u = u^k$. The energy equation
follows in the usual way (and without difficulties at the boundary) since
because of {\eqref{eq:first-k-divfree}}
\[ \int_G \frac{\partial u_i}{\partial x_{\alpha}} u_{\alpha} u_i \mathd x =
   \int_G \frac{\partial K}{\partial x_{\alpha}} u_{\alpha} \mathd x = 0
   \hspace{1em} \left( K = \frac{1}{2} u_i u_i \right) \]
and
\[ \int_G \frac{\partial^2 u_i}{\partial x_{\beta} \partial x_{\beta}} u_i
   \mathd x = - \int_G \frac{\partial u_i}{\partial x_{\beta}}  \frac{\partial
   u_i}{\partial x_{\beta}} \mathd x \hspace{1em} (u = u^k) . \]
It follows from {\eqref{eq:energy-equation}} that
\[ \int_G u_i u_i \mathd x = \lambda_1^2 + \cdots + \lambda_k^2 \hspace{1em}
   (u = u^k) \]
never increases. From this we conclude that any solution of the differential
system {\eqref{eq:first-k-ode}} begun at $t = 0$ exists for all $t = 0$
({\color{red} ???weird}).

The approximation procedure may very easily be interpreted formally in the
following manner. We think of both sides of the Navie-Stokes differential
equations and the solution $u$ formally as if they were expanded in the
orthonormal system of the fields $a^{\nu}$: $u = \lambda_{\nu} a^{\nu}$. We
then obtain purely formally a system of infinitely many differential equations
of first order for the infinitely many scalar Fourier coefficients $\lambda$.
Our $k$th step then simply consists of only considering the first $k$ of these
equations and setting all unknowns with indices $\nu > k$ to zero. The way in
which we subsequently prove our existence theorem simultaneously yields a
statement regarding the convergence properties of this simplest and most
natural approximation method.

We choose the initial values of the $\lambda_{\nu} (t)$ at $t = 0$ to be the
Fourier coefficients of the expansion of the given field $U (x)$ in the
$a^{\nu}$. While the solutions $\lambda (t)$ in the $k$th step generally
depend on $k$, these initial values are independent of them. By the assumption
that $U \in H (N)$ in $G$ and by the completeness lemma
\ref{lem:testfields-completeness}, we have
\begin{equation}
  \label{eq:initial-values-of-sequence} u_k (x, 0) \rightarrow U (x)
  \hspace{1em} \text{strongly in $G$} \hspace{1em} (k \rightarrow \infty) .
\end{equation}

\section{Proof of the Existence Theorem.}

We summarize the properties of the fields of the sequence which we will need
in the following:
\begin{enumeratealpha}
  \item Each $u^k (x, t)$ is twice continuously $x$-$t$-differentiable and
  divergence-free for $x \subset G$, $t > 0$.
  
  \item $u^k (x, t)$ vanishes if $x$ lies outside a compact subset of the
  $x$-region $G$ that only depends on $k$.
  
  \item $u^k (x, t)$ satisfies the equation {\eqref{eq:ns-weak-first-k}} ($t
  \geqslant 0$) and the equation {\eqref{eq:ns-tau-to-tauprime}} ($\tau' >
  \tau \geqslant 0$) in the $k$ cases {\eqref{eq:testfield-approx}} ($\nu = 1,
  2, \ldots, k)$.
  
  \item The integrals
  \[ \int_G u_i u_i \mathd x, \hspace{1em} \int_0^{} \int_G \frac{\partial
     u_i}{\partial x_{\beta}}  \frac{\partial u_i}{\partial x_{\beta}} \mathd
     x \mathd t \hspace{1em} (u = u^k (x, t)) \]
  remain beneath a bound which is independent of $k, t, T$.
  
  \item The initial values $u^k (x, 0)$ satisfy the limit relationship
  {\eqref{eq:initial-values-of-sequence}}.
\end{enumeratealpha}
d) follows immediately from the temporally integrated energy equation
{\eqref{eq:energy-equation}} in connection with
{\eqref{eq:initial-values-of-sequence}}.

{\tmem{First step}}. Each field $a^{\nu} (x)$ is continuous in $G$ and
different from zero only in a compact subset of $G$. If we apply the first
half of d) to the right hand side of {\eqref{eq:ns-weak-first-k}} ($a =
a^{\nu}$) by estimating the term linear in $u = u^k$ by means of the Schwarz
Inequality and the term quadratic in $u$ by means of an absolute bound for the
derivatives of $a$, we obtain the following: The right hand side of
{\eqref{eq:ns-weak-first-k}} ($a = a^{\nu}$, $u = u^k$, $k \geqslant \nu$) is
uniformly bounded for fixed $\nu$ for all $k$ and $t$. The same is true of the
left hand side
\[ \frac{\mathd}{\mathd t} \int_G a_i u_i \mathd x. \]
For fixed $\nu$, the time functions
\[ \int_G a_i^{\nu} (x) u_i^k (x, t) \mathd x \]
satisfy a Lipschitz condition for all $t \geqslant 0$ that is independent of
$k$. Furthermore, they remain uniformly bounded for all $t$ and $k$. So by a
well-known {\color{green} choice theorem} (Auswahlsatz) there exists for an
arbitary, fixed $\nu$ a sequence of integers $k'$ such that
\begin{equation}
  \label{eq:subseq-limit-existence} \lim_{k' \rightarrow \infty} \int_G
  a_i^{\nu} (x) u_i^k (x, t) \mathd x
\end{equation}
exists for any $t \geqslant 0$, in fact uniformly so in any finite
$t$-interval. The sequence of $k'$ depends of the index $\nu$, but we may pick
the sequence belonging to the index $\nu + 1$ as a subsequence of the previous
one. By means of a diagonal argument we may thus form a fixed sequence of
integers (which we will once again label as $k'$) for which the limit
statement above holds properly for any fixed $\nu = 1, 2, \ldots$. In the
sequel, we will operate on this sequence of $k'$.

{\tmem{Second step.}} We will now prove that the sequence of fields $u^{k'}
(x, t)$ converges weakly in the $x$-region $G$ for each fixed $t \geqslant 0$.
For the purposes of our proof, we now fix an arbitrary, fixed value $t_0$ of
$t$ and observe that by the first half of 5d) the sequence of these fields ($t
= t_0$) is weakly compact in $G$. The claim will be proven when we show that
that sequence may possess only a single weak limit field in $G$. Let $u^{\ast}
(x, t_0)$ be such a limit field and let $k''$ be a subsequence of the $k'$
(this subsequence will depend on $t_0$) such that
\[ \lim_{k'' \rightarrow \infty} \int_G w_i (x) u_i^{k''} (x, t_0) \mathd x =
   \int_G w_i (x) u_i^{\ast} (x, t_0) \mathd x \]
for each field $w (x)$ of class $H$ in $G$. In the case $w = a^{\nu}$, the
value of the right hand side is already fixed by the limit
{\eqref{eq:subseq-limit-existence}}. If $u^{\ast}$ and $u^{\ast \ast}$ are two
weak limit fields and if $v$ is their difference field, then
\[ \int_G a_i^{\nu} v_i \mathd x = 0 \]
for each $\nu$. By Definition \ref{def:h-n} the fields $u^{\ast}$, $u^{\ast
\ast}$ and thus also $v$ belong to class $H (N)$ in $G$. However, by Lemma
\ref{lem:testfields-completeness} the fields $a^{\nu}$ span the same field
space in $G$. From this we conclude
\[ \int_G v_i v_i \mathd x = 0 \]
and thus the claim.

Consequently, there is a field $u^{\ast}$ which is well-defined in $G$ for all
$t > 0$ such that
\begin{equation}
  \lim_{k' \rightarrow \infty} \int_G w_i (x) u_i^{k'} (x, t) \mathd x =
  \int_G w_i (x) u^{\ast}_i (x, t) \mathd x
\end{equation}
for each field $w (x)$ ($w \in H$ in G) and for each $t > 0$. The field
$u^{\ast}$ satisfies condition B) of the existence theorem \ref{the:existence}
at the end of Section \ref{sec:bc-vanish-ivp}. This follows from b) and the
second half of 5d) by applying Lemma \ref{lem:cylinder-limit-hprime}. One
easily proves that $u^{k'} \rightarrow u^{\ast}$ also holds weakly in $x$ and
$t$ ($0 < t < T$).

{\tmem{Third step}}. The proof that the field $u^{\ast} (x, t)$ satisfies
condition A) of the existence theorem. In each cylinder region $x \subset$G,
$0 < t < T$, $u^{\ast}$ belongs to class $H'$, which is, as we remarked, a
superclass of $H' (N)$ (and because of B) it also belongs to the latter
class). By the arguments in the first half of Section
\ref{sec:simplification-approximation} we only need to show that $u^{\ast}$
satisfies the equations {\eqref{eq:ns-tau-to-tauprime}} for every $a =
a^{\nu}$ and for all $\tau' > \tau > 0$. By c), $u = u^{\ast}$ satifies these
equations for the same $\tau, \tau'$ and for the first $k'$ fields $a^{\nu}$.
We now fix $\tau$, $\tau'$ and the index $\nu$ and pass to the limit $k'
\rightarrow \infty$. It is clear that on the left hand side of
{\eqref{eq:ns-tau-to-tauprime}} $u$ may be replaced by $u^{\ast}$. The same
ist true of the third integral on the right hand side (the first one is zero).
Consider that in
\[ \int_{\tau}^{\tau'} \left[ \int_G w_i (x) u_i^{k'} (x, t) \mathd x \right]
   \mathd t \]
the inner integral is a uniformly bounded function with respect to \ $k'$
because of the first half of d) and that we may apply a well-known Lebesguian
convergence theorem to the outer $t$-integral. It requires some deeper
thoughts that make use of the second half of d) to see that we may also
interchange the limit $k' \rightarrow \infty$ and the integration in the
second integral on the right hand side of {\eqref{eq:ns-tau-to-tauprime}}. For
this, we need the following theorem which we will prove later.

\begin{lemma}
  \label{lem:integral-2-convergence}Let a sequence of functions $f^k (x, t)$
  which are continuously $x$-differentiable for $x \subset G$, $0 < t < T$
  have the following properties: For each fixed $t$, $f^k$ belongs to class
  $N$. For each fixed $t$, the $f^k (x, t)$ converge weakly in $G$ to a
  function $f^{\ast} (x, t)$. The integrals
  \[ \int_G f^2 (x, t) \mathd x, \hspace{1em} \int_0^T \int_G f_{' i} f_{' i}
     \dxdt \hspace{1em} (f = f^k) \]
  remain uniformly bounded with respect to $t$ and $k$. Then the $f^k$
  converge strongly to $f^{\ast}$ on the $x$-$t$-region $x \subset Q G$, $0 <
  t < T$, where $Q$ is an arbitrary finite cuboid in $x$-space. In particular,
  the assertion holds for $G$ itself if $G$ is bounded.
\end{lemma}

Because of a), b), because of the result of the second step and because of d),
the assumptions of the lemma are satisfied for the components of the sequence
of fields $u^{k'} (x, t)$ for an arbitrary fixed $T$. Thus, it follows that
\[ \int_0^T \int_{Q G} (u_i - u_i^{\ast}) (u_i - u_i^{\ast}) \dxdt
   \hspace{1em} (u = u^{k'}) \]
goes to zero for $k' \rightarrow \infty$ if $Q$ is an arbitry finite cuboid of
$x$-space. We can thus justify the passing to the limit in the second integral
on the right hand side of {\eqref{eq:ns-tau-to-tauprime}} ($a = a^{\nu}$,
$\nu$ fixed). Recall that the factor $a$ of the integrand vanishes outside a
fixed compact subset $C$ of $G$. If we choose $Q \supset C$ and $T > \tau'$,
then for the integral
\[ \int_{\tau}^{\tau'} \int_{Q G} (a_{i, \alpha}) (u_{\alpha}) \dxdt
   \hspace{1em} (a = a^{\nu}, u = u^{k'}) \]
we have the following stuation. The first factor converges weakly in the area
of integration to $a_{i, \alpha} u_{\alpha}^{\ast}$, while the second one
converges strongly to $u_i^{\ast}$. As is well-known, this suffices to carry
out the passing to the limit under the integral sign. We have thus shown that
the field $u^{\ast}$ satisfies the equations {\eqref{eq:ns-tau-to-tauprime}}
for any field $a^{\nu} (x)$ and for all positive $\tau$, $\tau'$. The
condition A) of the existence theorem is thus verified except for the freedom
from divergence. This latter property, however, is trivially true, even for
any fixed $t > 0$.

To complete the proof of the existence theorem, we only need to show that the
initial condition C) is also satisfied. From the energy equation
{\eqref{eq:energy-equation}} follows
\begin{equation}
  \label{eq:energy-ic} \frac{1}{2} \int_G u_i u_i \mathd x|_0 = \frac{1}{2}
  \int_G u_i u_i \mathd x|_T + \int_0^T \int_G \frac{\partial u_i}{\partial
  x_{\beta}}  \frac{\partial u_i}{\partial x_{\beta}} \dxdt
\end{equation}
for each field $u$ of our sequence. The left hand side tends to
\[ \frac{1}{2} \int_G U_i U_i \mathd x \]
for $k' \rightarrow \infty$ because of
{\eqref{eq:initial-values-of-sequence}}. For $t = T$, the fields converge
weakly to $u^{\ast}$ in $G$. In an $x$-$t$-cylinder section, we have
\[ u_{i, \beta}^{k'} \rightarrow u_{i, \beta}^{\ast} \]
weakly because of Lemma \ref{lem:weak-deriv-weak-conv} and d). By applying
Lemma \ref{lem:l2-lsc}, {\eqref{eq:energy-ic}} implies the inequality
\[ \frac{1}{2} \int_G U_i U_i \mathd x \geqslant \frac{1}{2} \int_G u_i^{\ast}
   u_i^{\ast} \mathd x|_T + \mu \int_0^T \int_G u_{i, \beta}^{\ast} u_{i,
   \beta}^{\ast} \dxdt \]
for an arbitrary $T > 0$. In particular,
\[ \overline{\lim_{t \rightarrow 0}} \int_G u_i^{\ast} u_i^{\ast} \mathd x
   \leqslant \int_G U_i U_i \mathd x. \]
If we once again apply Lemma \ref{lem:l2-lsc} to this last inequality, we
recognize that the initial condition C) is satisfied, which is what we wanted
to show.

We will not go into detail on the question of strong convergence for a fixed
$t$.

\section{Proof of Lemma \ref{lem:integral-2-convergence}}

The lemma is closely related to the {\color{red} Rellich Choice Theorem}
(Auswahlsatz) and is proven similarly as well{\footnote{Cf.
{\tmname{Courant-Hilbert}}, l.c. footnote \ref{fn:bc-finite-kinetic}, p. 218.
In Rellich's Theorem, the boundedness of the $x$-integrals of the squares of
the derivatives is assumed. Our boundedness assumption merely concerns the
$x$-$t$-integral and is thus better adapted to the state of affairs in our
problem.

Leray proves and uses a lemma even closer to the {\color{red} Rellich Choice
Theorem} (Auswahlsatz) l.c. Footnote \ref{fn:formulation-xt}, p. 214, Lemma 2,
which, like this theorem, only works with the $x$-integral. Our convergence
proof is more direct.}}.

Let us note up front that the lemma, just like Rellich's Theorem, need not
hold for $G$ itself if $G$ is infinite. A counterexample is given by the case
where $G$ is the entire $x$-space and
\[ f^k (x, t) = f (x_1 + k, x_2, \ldots, x_n) \]
with $f$ belonging to $H'$ and $N$ in $G$. In this case, $f^{\ast} = 0$, but
there is no strong covnergence to zero{\footnote{We may thus only conclude the
strong convergence of the approximate fields $u (x, t)$ to $u^{\ast} (x, t)$
in the cylinder sections if $G$ is bounded. However, strong convergence is
clearly true for arbitrary $G$. Leray deduced it for his aprpoximations in the
case where $G$ is the entire $x$-space using complicated estimates of the
distribution of energy over $G$. We hope to come back to the stronger
convergence properties of our approximations at some later date.}}.

The proof of Lemma \ref{lem:integral-2-convergence} arises from Friedrichs'
Inequality: Let $Q$ be a finite cuboid in $x$-space. For any given
$\varepsilon > 0$, there exists a finite number of fixed functions
$\omega_{\nu} (x)$ which belong to $H$ in $Q$ such that the inequality
\[ \int_Q f^2 \mathd x \leqslant \sum_{\nu} \left[ \int_Q f \omega_{\nu}
   \mathd x \right]^2 + \varepsilon \int_Q f_{' i} f_{' i} \mathd x \]
is satisfied by any function $f (x)$ belonging to $H'$ in $Q${\footnote{The
$\omega_{\nu}$ may be assumed to be orthogonal in $Q$. The inequality then
represents an estimate of the difference in Bessel's inequality. You may find
the proof of the inequality in {\tmname{Courant-Hilbert}}, l.c. footnonte
\ref{fn:bc-finite-kinetic}, p. 218, Chap. VII, �3, Section 1. We may easily
convince ourselves that the proof that is given there in 2 dimensions also
works in $n$ dimensions. Friedrichs' Inequality does not hold for arbitrary
bounded regions.}}. For the proof of Lemma \ref{lem:integral-2-convergence},
we first note that for fixed $t$ the functions $f^k (x, t)$ of the lemma are
continuously differentiable in $G$ and of class $N$. If we define the
functions to be zero outside $G$, then this statement remains valid if we
relate it to the entire $x$-space instead of to $G$. In particular, any of the
functions on any finite cuboid $Q$ of $x$-space belongs to class $H'$. The
extension of the functions and the last statement were made possible by the
assumption of membership in lcass $N$. This is however the only place where
this assumption is used. We now fix a cuboid $Q$ and a number $\varepsilon >
0$ arbitrarily and pick the finitely many auxiliary functions $\omega_{\nu}
(x)$ such that Friedrichs' Inequality holds in $Q$. We apply it to the
functions
\begin{equation}
  \label{eq:friedrichs-victims} f (x, t) = f^k (x, t) - f^l (x, t),
\end{equation}
which surely belong to $H'$ in $Q$, for fixed $t$. By integration in $t$, we
conclude that all the functions {\eqref{eq:friedrichs-victims}} satisfy the
inequality
\begin{equation}
  \label{eq:friedrichs-difference} \int_0^T \int_Q f^2 \dxdt \leqslant
  \sum_{\nu} \int_0^T \left[ \int_Q f \omega_{\nu} \mathd x \right]^2 \mathd t
  + \varepsilon \int_0^T \int_Q f_{' i} f_{' i} \dxdt .
\end{equation}
By assumption (weak convergence for fixed $t$), we have
\[ \lim_{k \rightarrow \infty, l \rightarrow \infty} \int_Q f \omega_{\nu}
   \mathd x = 0 \]
for each fixed $t$. Because of the boundedness assumption (first half),
furthermore the function of $t$
\[ \int_Q (f^k - f^l) \omega_{\nu} \mathd x \]
remains uniformly bounded w.r.t. $k$, $l$. Thus the first term on the right
hand side in {\eqref{eq:friedrichs-difference}} tends to zero for $k
\rightarrow \infty$, $l \rightarrow \infty$. By assumption, the factor of
$\varepsilon$ for the functions {\eqref{eq:friedrichs-victims}} remains below
a fixed bound. But
\[ \overline{\lim}_{k \rightarrow \infty, l \rightarrow \infty} \int_0^T
   \int_Q (f^k - f^l)^2 \dxdt \leqslant c \varepsilon \]
implies strong convergence of our our sequence in the $x$-$t$-region $x
\subset Q$, $0 < t < T$, since $\varepsilon$ was arbitrary. We easily obtain
that the limit function is the function $f^{\ast} (x, t)$ mentioned in the
statement of the lemma. Thus, Lemma \ref{lem:integral-2-convergence} is
proven.

\end{document}
